\documentclass[UTF8]{article}
    \usepackage{ctex}
    \usepackage{graphicx}
    
    \usepackage{geometry}
    %\geometry{left=2.5cm,right=2.5cm,top=2.5cm,bottom=2.5cm}
    \geometry{scale=0.8}
    
    \usepackage[utf8]{inputenc}
    \usepackage{amssymb}
    \usepackage{amsmath}
    
    \usepackage{hyperref} %支持书签%
    %\hypersetup{hidelinks}
    \hypersetup{colorlinks=true,linkcolor=black}
    
    %\usepackage[colorlinks,linkcolor=red,anchorcolor=blue,citecolor=green]{hyperref}
    
    \usepackage{enumerate}
    
    
    \title{初中问题}
    \author{yinqingwang }
    \date{July 2018}
    
    %% document start 
    \begin{document}
        \maketitle
        \tableofcontents

    \section{示例问题}
    \subsection{趣题}

    \section{数学问题}
    \subsection{代数}

    \begin{enumerate}
    %\newpage
    \item Prove that 
    
    $$ \frac{(a+b-kc)^2}{(a-c)(b-c)} + \frac{(b+c-ka)^2}{(b-a)(c-a)} + \frac{(c+a-kb)^2}{(a-b)(c-b)} = (k+1)^2  $$

    \noindent \textbf{Solution.} 
    \begin{align*}
        \Longleftrightarrow & \frac{(a+b-kc)^2(a-b)}{(a-b)(b-c)(a-c)} + \frac{(b+c-ka)^2(b-c)}{(a-b)(b-c)(a-c)} + \frac{(c+a-kb)^2(c-a)}{(a-b)(b-c)(a-c)} = (k+1)^2  \\
        \Longleftrightarrow & (a+b-kc)^2(a-b) + (b+c-ka)^2(b-c) + (c+a-kb)^2(c-a) = (k+1)^2(a-b)(b-c)(a-c) \\
        \Longleftrightarrow & [(a+b)^2(a-b)-2kc(a+b)(a-b)+k^2c^2(a-b)] + \\
        & [(b+c)^2(b-c)-2ka(b+c)(b-c)+k^2a^2(b-c)] + \\
        & [(c+a)^2(c-a)-2kb(c+a)(c-a)+k^2b^2(c-a)] = (k+1)^2(a-b)(b-c)(a-c) \\
        \Longleftrightarrow & (a^2b+b^2c+c^2a-ab^2-bc^2-ca^2) + \\
        & 2k(a^2b+b^2c+c^2a-ab^2-bc^2-ca^2) + \\
        & k^2(a^2b+b^2c+c^2a-ab^2-bc^2-ca^2) = (k+1)^2(a-b)(b-c)(a-c) \\
        \Longleftrightarrow & (k^2+2k+1)(a^2b+b^2c+c^2a-ab^2-bc^2-ca^2) = (k+1)^2(a-b)(b-c)(a-c) \\
        \Longleftrightarrow & (k+1)^2(a-b)(b-c)(a-c) = (k+1)^2(a-b)(b-c)(a-c)
    \end{align*}

    \item 化简: $ \sqrt[3]{2+\sqrt{5}} $

    \noindent \textbf{Solution 1.}  Let 
    \[
     \begin{aligned}
        u = \sqrt[3]{2+\sqrt{5}} \, > 0 \\
        v = \sqrt[3]{2-\sqrt{5}} \, < 0
    \end{aligned}
    \Longrightarrow 
    \left \{
    \begin{aligned}              
        u^3 + v^3 = 4 \\
        u \cdot v = -1
    \end{aligned}
    \Longrightarrow 
    (u+v)\left[(u+v)^2 - 3uv\right] = 4
    \right.
    \]
    Let 
    \[
    t = u + v
    \Longrightarrow t^3 + 3t - 4 = 0
    \Longrightarrow (t-1)(t^2+t+4) = 0
    \Longrightarrow t = 1
    \]
    So,
    \begin{align*}
        \left\{
            \begin{aligned}
                u + v = 1 \\
                u \cdot v = -1
            \end{aligned}
        \right.
        \Longrightarrow  u = \sqrt[3]{2+\sqrt{5}} = \frac{1+\sqrt{5}}{2} \\
        \blacksquare
    \end{align*}
 
    \noindent \textbf{Solution 2.}  Let $$ \sqrt[3]{2+\sqrt{5}} = (a + b \sqrt{5}) $$ \\
    So, 
    \begin{align*}
        & 2 + \sqrt{5} = (a+b\sqrt{5})^3 = (a^3 + 15ab^2) + (3a^2b + 5b^3)\sqrt{5} \\
        \Longrightarrow & \left\{
            \begin{aligned}
                a^3 + 15ab^2 = 2 \\
                3a^2b + 5b^3 = 1
            \end{aligned}
            \Longrightarrow a^3 + 15ab^2 = 2 \cdot (3a^2b + 5b^3)
            \Longrightarrow (\frac{a}{b})^3 -6(\frac{a}{b})^2 + 15(\frac{a}{b}) -10 = 0
            \right. \\
            \Longrightarrow & \frac{a}{b} = 1 \  , \  a = b = \frac{1}{2} \\
            & \sqrt[3]{2 + \sqrt{5}} = \frac{1+\sqrt{5}}{2} \\
            \blacksquare
        \end{align*}

        \item 解方程: $$ \sqrt[3]{5+x} +  \sqrt[3]{4-x} = 3 $$
        
        \noindent \textbf{Solution. }  Let
        \begin{align*}
           & u = \sqrt[3]{5+x} \ , \ v = \sqrt[3]{4-x} \\
            \Longrightarrow &
            \left \{
                \begin{aligned}
                u + v = 1 \\
                u^3 + v^3 = 9
                \end{aligned}
                \Longrightarrow u \cdot v = 2
            \right. \\
            \Longrightarrow & u,v = 1,2 \\
            \Longrightarrow & x = -4, 3 \\
            \blacksquare 
        \end{align*}

        %\item   

    \end{enumerate}

\end{document}