\documentclass[UTF8]{article}
\usepackage{ctex}
\usepackage{graphicx}

\usepackage{geometry}
%\geometry{left=2.5cm,right=2.5cm,top=2.5cm,bottom=2.5cm}
\geometry{scale=0.8}

\usepackage[utf8]{inputenc}
\usepackage{amsmath}

\title{中学数学问题}
\author{yinqingwang }
\date{July 2016}
\begin{document}
	\maketitle
	\tableofcontents
\newpage

\section{练习题}
\noindent 1. 有单位圆内接五边形,此五边形的边长以及对角线共十条线段, 求此每线段的平方之和的最大值。\\
\noindent \textbf{Solution.} \\
设单位圆的圆心为坐标系的原点, 此五边形的五个顶点的坐标为 $(\cos\theta_i,\sin\theta_i)$, $0\le \theta_i < \theta_j < 2\pi, 1 \le i < j \le 5$. 那么所求为,
 $$ S = \sum_{i=1}^{4}\sum_{j=i+1}^{5}[(\cos\theta_j-\cos\theta_i)^2+(\sin\theta_j-\sin\theta_i)^2] = \sum_{i=1}^{4}\sum_{j=i+1}^{5}(2-2\cos(\theta_j-\theta_i))$$


\section{2017年}

\subsection{2017北京大学数学科学学院秋令营10月13日}
\subsubsection{第3题}
\noindent 3.给定素数$p$, 正整数$n,a$, 其中$(a,p)=1$, 证明: 存在无穷多个正整数k,使得 $p^n|k^k-a$.\\
\textbf{Proof.}\\
对$n$归纳证明. \\
记欧拉函数为$\phi (n)$, 则 : $\phi (p^n) = (p-1)p^{n-1}$ \\
(1) 当$n=1$时, 取$k$满足如下条件: $k \equiv a \pmod p ,\quad	k \equiv 1 \pmod {p-1}$, \\
那么由中国剩余定理解出满足条件的 $k = b + m\cdot p(p-1) , 0<b < p(p-1)$, 其中: 
\begin{eqnarray*}
&	b \equiv a \pmod p ,\quad	b \equiv 1 \pmod {p-1} \\
&	\Longrightarrow a^b \equiv a^{1+h\cdot (p-1)} \equiv a \pmod p
\end{eqnarray*}
从而, 有 $(k,p)=1$:
$$k \equiv a \pmod p \Longrightarrow k^k \equiv a^k \equiv a^b \equiv a \pmod p$$ 
(2) 假设 $n=t$时,命题成立, 即存在$u$, $(u,p)=1$, 并且满足: $u^u \equiv a \pmod {p^t}$, 记 $u^u = l\cdot p^t + a$,\\
当 $n=t+1$时, 取 $k=u + m\cdot \phi (p^{t+1})$, 那么$(k,p)=1$, 从而$(k,p^{t+1}) =1$, 于是: 
\begin{eqnarray*}
	& k \equiv u+m\cdot \phi(p^{t+1}) \equiv u + m\cdot p^{t+1} - m\cdot p^t \equiv u-m\cdot p^t \pmod{p^{t+1}}\\
	&\Longrightarrow  k^k \equiv (u-m\cdot p^t)^k \equiv (u-m\cdot p^t)^u \equiv u^u -C_u^1\cdot u^{u-1} m\cdot p^t \equiv u^u - u^u\cdot m\cdot p^t \\
	& \equiv u^u(1-m\cdot p^t) \equiv (l\cdot p^t +a)(1-m\cdot p^t) \equiv a + (l-a\cdot m)p^t \pmod{p^{t+1}}
\end{eqnarray*}
因为 $(a,p)=1$, 所以存在$m$, 满足: $l-am=s\cdot p$, \\
并且满足条件的$m$是无穷多个,从而$k$也是无穷多个, 于是:
$$ k^k \equiv a + s\cdot p^{t+1} \equiv a \pmod {p^{t+1}}$$
由归纳假定, 对任意的$n$, 存在无穷多个$k$, $(k,p)=1$ 满足: $p^n|k^k-a$.

\newpage
\subsection{2017清华大学金秋营10月13号}
\subsubsection{第2题}
\noindent 2.设连续函数 $f: (0,+\infty) \longrightarrow (0,+\infty)$, 满足: $f(\frac{x+y}{2}) = f(\sqrt{xy}); (x,y>0)$ , 求出所有满足条件的 $f(x)$, 并证明。\\
\textbf{Solution.} \\
(1) 令$y=\frac{1}{x}$, 于是:
$$f\left (\frac{1}{2}(x+\frac{1}{x})\right) = f(1)$$
记: 
$$z=\frac{1}{2}(x+\frac{1}{x}) \ge 1 \Longrightarrow f(z) = f(1) $$
(2) 令$x=1+a,y=1-a , 0<a<1$,则有:
$$f(1) = f(\sqrt{1-a^2})$$
记:  $z=\sqrt{1-a^2} \in (0,1)$, 则有:
$$f(z) = f(1)$$

综上, 得到: $f(x)=f(1) = \text{常数}, x\in (0,+\infty).$
\subsubsection{第3题}
\noindent 3.设 $m,n\in N, m \le n$, 证明: $\sum_{i=0}^{m}C_n^i < \left(\frac{3n}{m}\right)^m$
\\
\textbf{证:}\\
\begin{eqnarray*}
C_n^i=\frac{n!}{i!(n-i)!} = \frac{n(n-1)\cdots(n-i+1)}{i!} < \frac{n^i}{i!} = \frac{m^i}{i!}\cdot \left(\frac{n}{m}\right)^i < \left( \frac{n}{m}\right)^m \cdot \frac{m^i}{i!} \\
\\
\sum_{i=0}^{m}C_n^i < \sum_{i=0}^{m}\left( \frac{n}{m}\right)^m \cdot \frac{m^i}{i!} = \left( \frac{n}{m}\right)^m \cdot \sum_{i=0}^{m}\frac{m^i}{i!} < \left( \frac{n}{m}\right)^m \cdot e^m = \left( \frac{e\cdot n}{m}\right)^m < \left( \frac{3n}{m}\right)^m
\end{eqnarray*}

显然有: $$\sum_{i=0}^{m}C_n^i  = \sum_{i=0}^{m}C_{n-1}^i + \sum_{i=0}^{m-1}C_{n-1}^i $$ \\
当$m=n$时, 不等式显然成立。 下面假设 $1 \le m<n$. 对$n$进行归纳证明。\\
(1) 当$n=2$时, 有$m=1$, 不等式显然成立。 \\
(2) 假设当$n=k$时, 不等式 $\sum_{i=0}^{m}C_k^i < \left(\frac{3k}{m}\right)^k$ 成立, 则当$n=k+1$时, 有:\\
\begin{eqnarray*}
	\sum_{i=0}^{m}C_{k+1}^i = C_{k+1}^0 + \sum_{i=1}^{m}(C_k^i + C_k^{i-1}) = \sum_{i=0}^{m}C_k^i + \sum_{i=0}^{m-1}C_k^i < \left(\frac{3k}{m}\right)^k + \left(\frac{3k}{m-1}\right)^k \\
	\Longleftrightarrow \left(\frac{3k}{m}\right)^k + \left(\frac{3k}{m-1}\right)^k < \left(\frac{3(k+1)}{m}\right)^{k+1} = \frac{3(k+1)}{m} \cdot \left(\frac{3(k+1)}{m}\right)^{k}	\\
	\Longleftrightarrow 1+\left(\frac{m}{m-1}\right)^k < \frac{3(k+1)}{m} \left(\frac{k+1}{k}\right)^{k}
\end{eqnarray*}

\subsubsection{第5题}
\noindent 5. 给定奇素数$p$, 求集合$\left\{(x,y)\mid x^2+y^2 \equiv a \pmod{p}; x,y\in \{0,1,2,\cdots ,(p-1)\} \right\}$ 的元素个数。
\\
\textbf{Solution.} \\
首先, $a \ne 0 \pmod p$时, 方程$uv\equiv a \pmod p $的解$(u,v)$的数目是 $p-1$,为简单记,假定$a\in\{0,1,2,...,p-1\}$ .\\
记 某个元$b$模$p$的逆元$c=b^{-1}$, 也即$bc\equiv b\cdot b^{-1}\equiv1\pmod{p}$\\
(1) 当$p\equiv1 \pmod{4}$时,由欧拉判别法,若$b$是模$p$的平方剩余, 则$-b$也是模$p$的平方剩余, 于是平方剩余是以$\pm b$的形式成对出现。从而$-1$是模p的平方剩余,可记$z^2\equiv -1 \pmod{p}$, 

(i) 如果$a = 0 $, 那么$x^2\equiv -y^2 \pmod{p}$, 由于平方剩余以$\pm b$形式成对出现,因此$x=y=0$,或者$x,y\ne 0$以$2(p-1)$组出现,
此时解的总数为: $1+2(p-1)=2p-1$

(ii) 如果$a\ne 0 $, 那么有:
\begin{eqnarray*}
	&x^2+y^2 \equiv a \pmod{p} & \Longrightarrow x^2 - (-1)y^2 \equiv a \pmod{p}\\
	&\Longrightarrow x^2 - (zy)^2 \equiv a \pmod{p} & \Longrightarrow (x-zy)(x+zy) \equiv a \pmod{p}\\
	&\Longrightarrow u\cdot v \equiv a \pmod{p}
\end{eqnarray*}
其中,$(u,v)$ 和 $(x,y)$一一对应.
\begin{eqnarray*}
	u=x-zy,v=x+zy \\
	\Longleftrightarrow x=(u+v)2^{-1}, y=(u-v)2^{-1}z^{-1}
\end{eqnarray*}
不同的$(u,v)$一共有$p-1$组, 所以不同的解$(x,y)$的数目也是 $p-1$.
\\
(2) 当$p\equiv3 \pmod{4}$时,由欧拉判别法, 如果$b$是平方剩余, 那么$-b$是平方非剩余,

(i) 如果$a = 0 $, 那么$x^2+y^2 \equiv 0 \pmod p$,只有一组解: $(x,y)=(0,0)$

(ii) 如果$a\ne 0 $, 当$y$ 依次取遍 $\{0,1,2,...,p-1\}$时, 考虑$x$的可能取值时, 对每个$y$,下述方程只有一个有解:\\
\begin{eqnarray*}
	x^2 \equiv a - y^2 \pmod p \\
	x^2 \equiv y^2 - a \pmod p \\
\end{eqnarray*}
因为所有的$y$的数目是$p$, 所以上述两个方程给出的所有解$(x,y)$的数目是$2p$, 但是方程:\\
$$x^2 \equiv y^2 - a \pmod p \Longleftrightarrow x^2 - y^2 \equiv - a \equiv b\pmod p \Longleftrightarrow (x+y)(x-y) \equiv b \pmod p$$
解的数目是 $p-1$组, 故 $x^2 \equiv a - y^2 \pmod p$,也即$x^2 +  y^2\equiv a  \pmod p$ 的解的数目是: $2p-(p-1) = p+1$ 。\\
注: 如果$y^2 \equiv a \pmod p$,那么两个方程同时给出解$x=0$, 此时不同的$y$只有2个,所以总的解数目还是$2p$。\\
\\
综上, 解数目$(0 \le a < p-1)$:
$$ =\left\{
\begin{aligned}
2p-1, & \quad a = 0 , p \equiv 1 \pmod 4 \\
1, &\quad  a = 0 , p \equiv 3 \pmod 4 \\
p-1, &\quad  a\ne 0 , p \equiv 1 \pmod 4 \\
p+1, &\quad  a\ne 0 , p \equiv 3 \pmod 4 \\
\end{aligned}
\right.
$$

\section{Conclusion}	
	这是一个CTEX的utf-8编码例子,{\kaishu 这里是楷体显示},{\songti 这里是宋体显示},{\heiti 这里是黑体显示},{\fangsong 这里是仿宋显示},{\lishu 这里是隶书显示},{\youyuan 这里是幼圆显示}。
\end{document}