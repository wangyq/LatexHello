\documentclass[UTF8]{article}
\usepackage{ctex}
\usepackage{graphicx}

\usepackage{geometry}
%\geometry{left=2.5cm,right=2.5cm,top=2.5cm,bottom=2.5cm}
\geometry{scale=0.8}

\usepackage[utf8]{inputenc}
\usepackage{amssymb}
\usepackage{amsmath}

\usepackage{hyperref} %支持书签%
%\hypersetup{hidelinks}
\hypersetup{colorlinks=true,linkcolor=black}

%\usepackage[colorlinks,linkcolor=red,anchorcolor=blue,citecolor=green]{hyperref}



\title{中学数学问题}
\author{yinqingwang }
\date{July 2016}
\begin{document}
	\maketitle
	\tableofcontents
\newpage

\section{初中数学}
\subsection{一元二次方程}
\noindent 1. 已知实数 $a \ge b \ge c$, 并且满足: $a+b+c=0, a^2+b^2+c^2=1$, 求$a+b$的取值范围。 \\
\noindent \textbf{Solution.} \\
由已知, $ -c = a + b \ge 2c$, 故 $  c \le 0 $ \\
将 $c$ 看成常量, 易得 $a,b$为方程 : $x^2 + (x+c)^2 + c^2 -1 = 2x^2 + 2cx + 2c^2 -1 =0$ 的两根。
\begin{align*}
\Delta = (2c)^2 - 4\times 2 \times (2c^2-1) = 4(2-3c^2)\ge 0 \\
\Rightarrow  c^2 \le \frac{2}{3} \\
\\
x =a,b = \frac{-2c \pm \sqrt{\Delta }}{2\times 2} \ge c \\
\Rightarrow c^2 \ge \frac{1}{6} \\
-\sqrt{\frac{2}{3}} \le c \le -\sqrt{\frac{1}{6}} \\
\sqrt{\frac{1}{6}} \le a+b=-c \le \sqrt{\frac{2}{3}}
\end{align*}
$\blacksquare$ \\

\noindent 2. 已知 $ m \ne 0$ , 对任何的 $x\ge4$ 下述不等式都成立,求$m$的取值范围。 
$$ \frac{m^2\cdot x -1}{mx+1} < 0 $$

\noindent \textbf{Solution.} \\
分式变为乘积后为二次方程。 易得 $m<0$ 。两个根 $x_1,x_2 <4$ ,可以解得m的取值范围。 \\
$m<-1 , \quad -1/4<m<0 \quad \blacksquare$

\section{练习题}

\subsection{不等式}
\noindent 1. For natural number $n=1,2,3, ...$, prove that  $$ \sqrt[n]{n} \le 1 + \sqrt{\frac{2}{n}}$$  \\
\noindent \textbf{Proof.} \\
First , it is clear that $n < 2^n \Longrightarrow 1 < \sqrt[n]{n} < 2 = \sqrt[n]{2^n}$ \\
Let $ a = \sqrt[n]{n} -1 $,   $ 0<a<1 $ , which gives 
\begin{align*}
n & = (1+a)^n = 1 + na + \frac{n(n-1)}{2} \cdot a^2 + \cdots \\
& > na + \frac{n(n-1)}{2} \cdot a^2 = \frac{n^2}{2} a^2 + na - \frac{1}{2}na^2 = \frac{n^2}{2} a^2 + na(1-\frac{1}{2}a) \\
& > \frac{n^2}{2} a^2 \\
& \Longrightarrow a < \sqrt{\frac{2}{n}} \\
& \Longrightarrow \sqrt[n]{n} < 1 + \sqrt{\frac{2}{n}} 
\end{align*}
$\blacksquare$ \\

\subsection{复数相关}
\noindent 1. 有单位圆内接的任意五边形,对五边形的五个顶点, 连接每两个顶点得到一条线段,求这些线段的平方之和的最大值。\\
\noindent \textbf{Solution.} \\
以单位圆的圆心为坐标系的原点建立坐标系, 五边形的五个顶点以复数分别表示为: $z_i, \quad i=1,2,3,4,5$, \\
那么 $|z_i|=1$, 记所求的值为 $S$, 则有:
\begin{align*}
2S &= \sum_{i=1}^{5}\sum_{j=1}^{5}|z_i-z_j| = \sum_{i=1}^{5}\sum_{j=1}^{5}(z_i-z_j)(\bar{z_i}-\bar{z_j}) =  \sum_{i=1}^{5}\sum_{j=1}^{5}(|z_i|^2+|z_j|^2-z_i\bar{z_j} - \bar{z_i}z_j) \\
&=\sum_{i=1}^{5}\sum_{j=1}^{5}(2-z_i\bar{z_j} - \bar{z_i}z_j)=2\times5\times5-\sum_{i=1}^{5}\sum_{j=1}^{5}(z_i\bar{z_j} + \bar{z_i}z_j) = 50 - (\sum_{i=1}^{5}z_i\cdot \sum_{j=1}^{5}\bar{z_j} + \sum_{i=1}^{5}\bar{z_i}\cdot \sum_{j=1}^{5}z_j) \\
&= 50 - 2 \cdot \sum_{i=1}^{5}z_i \cdot \sum_{i=1}^{5}\bar{z_i} = 50 - 2 \left|\sum_{i=1}^{5}z_i \right|^2 \le 50 \\
S & \le 25
\end{align*}
等号成立当且仅当 $\sum z_i = 0$, 也即多边形为正五边形时,达到最大值 = 25。$\blacksquare $\\


\noindent 设单位圆的圆心为坐标系的原点, 此五边形的五个顶点的坐标为 $(\cos\theta_i,\sin\theta_i)$, $0\le \theta_i < \theta_j < 2\pi, 1 \le i < j \le 5$. 那么所求为,
 $$ S = \sum_{i=1}^{4}\sum_{j=i+1}^{5}[(\cos\theta_j-\cos\theta_i)^2+(\sin\theta_j-\sin\theta_i)^2] = \sum_{i=1}^{4}\sum_{j=i+1}^{5}(2-2\cos(\theta_j-\theta_i))$$

\subsection{函数方程}
\noindent 1. 已知 $f(f(x)) = -x$, 并且 $x, f(x) $ 为整数, 求其中一个 $f(x)$ 的表达式。 \\
\noindent \textbf{Solution.} \\
由已知, 有: 
\begin{align*}
f(f(f(x))) &= f(-x) \\
f(f(f(x))) &= -f(x) \\
\Longrightarrow  f(-x) & = -f(x) \\
\Longrightarrow f(0) & = 0
\end{align*}
也即映射 $f(x)$ 的状态为: $ x\rightarrow f(x)\rightarrow  -x\rightarrow -f(x)\rightarrow x$, \\
从而, 满足条件的 $f(x)$ 解有无穷多, 下面是其中一个解:
$$
f(n) = \left\{ 
\begin{aligned}
0,  \quad n=0 \\
n+1, \quad n>0\text{奇数}\\
-n+1,  \quad n>0\text{偶数}\\
n-1,  \quad n<0\text{奇数}\\
-n-1,  \quad n<0\text{偶数}
\end{aligned}
\right.
$$

或者: 
$$
\left. 
\begin{aligned}
f(2k-1) = 2k , \quad k \in Z^+ \\
f(2k) = 1-2k , \quad k \in Z^+ \\
f(0) = 0 \\
f(-x) = - f(x) 
\end{aligned}
\quad \right\}
$$

\subsection{概率相关}
\noindent 1、甲和乙下棋,一局中甲获胜的概率是2/3,乙获胜的概率是1/3。一方比另一方多赢两局算赢,求甲获胜的概率? \\
\noindent \textbf{Solution.} \\
记事件A: 一局中甲胜, 事件B: 一局中乙胜, 事件C: 最终甲胜\\
于是有: 
$$P(C) =P(\{AB,BA\} \cup \{AB,BA\} \cdots \cup AA) = (\frac{2}{3})^2 \sum_{n=0}^{\infty}(\frac{4}{9})^n = \frac{4}{5}$$
$\blacksquare$

\noindent 2.将一条给定的线段分成三段,求这三段能够拼成三角形的概率。 \\
\noindent \textbf{Solution.} \\
不失一般性, 设线段长度为1. 分别记三条线段长度为X,Y,Z. \\
三条线段取法为: 先随机取一段X, 然后从剩下的1-X中随机取一段Y, 最后的线段为Z. 于是随机变量为X,Y, 并且Z=1-X-Y. \\
由取法知道:
\begin{align*}
f_x(x) & = 1, \quad 0<x<1 \\
f_{Y|X}(y|x) & = \frac{1}{1-x}, \quad 0<y<1-x \\
\text{X,Y,Z构成三角形} & \Longrightarrow \text{区域D:}0<X<\frac{1}{2}, \quad 0<Y<\frac{1}{2},\quad \frac{1}{2}<X+Y<1
\end{align*}
于是有: 
\begin{align*}
P(\{\text{X,Y,Z构成三角形}\}) & = \iint_{D} f(x,y)\mathrm{d}x\mathrm{d}y  \\
 & = \iint_{D} f_x(x) f_{Y|X}(y|x) \mathrm{d}x\mathrm{d}y \\
 & = \ln2 - \frac{1}{2}= 0.193
\end{align*}

\noindent 3. 设A=n, B=m 是大于0的整数, 抛出一个硬币,出现正反面的概率相等,如果出现正面A减1,出现反面B减1。求A先减到0的概率。 \\
\noindent \textbf{Solution.} \\
设随机变量 $X=$ A变为0时, 正面出现的次数, $Y=$ A变为0时, 反面出现的次数。 那么有: \\
\begin{align*}
P(\{ X=n, Y<m \})  &= P(\{ X=n, Y=0 \} \cup \{ X=n, Y=1 \} \cup \cdots \{ X=n, Y=m-1 \} )   \\
&= (\frac{1}{2})^n \cdot \sum_{i=0}^{m-1} C_{n+i-1}^{n-1} (\frac{1}{2})^i  \\
\blacksquare 
\end{align*}

\noindent 4. 1) 假设随机变量X取0到正无穷之间的值,其数学期望E[X]为1,求P$\{X \geq 8\}$的概率最大是多少。\\
2) 设随机变量X服从“对数正态分布”(常用于个人收入分布模型),已知期望值为e,标准差为1,求 
P$\{X>8e\}$的概率值。 (对数正态分布(logarithmic normal distribution)是指一个随机变量的对数服从正态分布,则该随机变量服从对数正态分布。)\\
\noindent \textbf{Solution.}  \\
1)
\begin{align*}
	E[X] & = \int_{0}^{+\infty} {xf(x)dx} = \int_{0}^{a} {xf(x)dx} + \int_{a}^{+\infty} {xf(x)dx} \geq \int_{a}^{+\infty} {xf(x)dx} \geq \int_{a}^{+\infty} {af(x)dx} = a\cdot \int_{a}^{+\infty} {f(x)dx} \\
	& = a\cdot P\{X\geq a\} \\
	\Longrightarrow & P\{X\geq a\} \le \frac{E[X]}{a} \\
	\Longrightarrow & P\{X\geq 8\} \le \frac{E[X]}{8} = \frac{1}{8} = 0.125
\end{align*}
2) 记 $Y=\ln X \sim N(\mu,  \sigma^2) $ , 则 $Y$ 服从正态分布。\\
\begin{align*}
	E(X) & = e^{\mu + \sigma^{2}/2} = e, \quad \quad \quad D(X)=(e^{\sigma^2}-1)\cdot e^{2\mu+\sigma^2} = 1 \\
	\Longrightarrow & \mu = \ln (E(X)) - \frac{1}{2} \ln \left(1+\frac{D(X)}{E(X)^2} \right ) = 2 - \frac{1}{2}\ln(1+e^2) \\
	& \sigma^2 = \ln \left( 1+ \frac{D(X)}{E(X)^2}\right) = \ln (1+e^2) - 2 \\
	So, \\
	P\{X > 8e \} &= P\{Y=\ln X > 1 + \ln8 \} = 1- P\{Y=\ln X \le 1 + \ln8 \} \\
	& = 1 - \Phi(\frac{1+\ln8-2+\frac{1}{2}\ln(1+e^2)}{\sqrt{\ln(1+e^2)-2}}) = 1-\Phi(6.01484) \approx 0
\end{align*}
$\blacksquare$

\subsection{数列和级数}

\noindent 1. 判断级数$S_n$的敛散性: $$S_n = \sum_{i=1}^{n}\frac{1}{i(n-i+1)} = \frac{1}{1\cdot n} + \frac{1}{2(n-1)} + \frac{1}{3(n-2)}+ \cdots + \frac{1}{n \cdot 1}$$ \\
\noindent \textbf{Solution.} \\
首先, 对于 $k>1$ 易知: 
\begin{align*}
\int_{k-1}^{k}\frac{1}{x}dx = \ln k - \ln(k-1) \\
\int_{k-1}^{k}\frac{1}{x}dx > \int_{k-1}^{k}\frac{1}{k}dx = \frac{1}{k}\int_{k-1}^{k}dx = \frac{1}{k} \\
\Longrightarrow \frac{1}{k} < \ln k - \ln (k-1) \\
\Longrightarrow 1+\frac{1}{2} + \frac{1}{3} + \cdots + \frac{1}{n} < 1+ \ln n
\end{align*}
所以, 
\begin{align*}
S_n & = \sum_{i=1}^{n}\frac{1}{i(n+1-i)} = \frac{1}{n+1}\sum_{i=1}^{n}\left(\frac{1}{i}+\frac{1}{n+1-i}\right) = \frac{2}{n+1} \sum_{i=1}^{n}  \frac{1}{i} \\
& < \frac{2}{n+1}(1+\ln n) = \frac{2}{n+1} + \frac{2}{n+1} \ln n = \frac{2}{n+1} + 2 \cdot \ln \left(\sqrt[n+1]{n}\right) \\
\Longrightarrow & \lim_{x \to \infty}S_n = 0 + 2\cdot \ln 1 = 0 
\end{align*}
$\blacksquare$

\subsection{微积分}
\noindent 1. 设函数 $f(t), g(t) $ 定义于 $[0,+\infty), \quad f(t)>0, \quad g(t) >0$ 且存在常数 $C_0,C_1$ 有:  $$\int_{0}^{t}g(s)ds \leq C_0$$  $$f(t) \leq C_1 \cdot \left(2+\int_{0}^{t}g(s)f(s)ds\right) \cdot \ln(2+\int_{0}^{t}f(s)g(s)ds)$$ 
证明: 存在常数 $M>0$, 使得 $f(t) \leq M, \quad t\in [0,+\infty)$ \\
\noindent \textbf{Solution.} \\
由已知, 得到:
\begin{align*}
 \frac{f(t)}{(2+\int_{0}^{t}g(s)f(s)ds ) \cdot \ln(2+\int_{0}^{t}f(s)g(s)ds)} \leq C_1 \\
\Longrightarrow  \frac{f(t)g(t)}{(2+\int_{0}^{t}g(s)f(s)ds ) \cdot \ln(2+\int_{0}^{t}f(s)g(s)ds)} \leq C_1\cdot g(t) \\
\Longrightarrow \int_{0}^{t} \frac{f(t)g(t)}{(2+\int_{0}^{t}g(s)f(s)ds ) \cdot \ln(2+\int_{0}^{t}f(s)g(s)ds)} dt \leq \int_{0}^{t}C_1\cdot g(t) dt \\
\Longrightarrow \ln \ln (2+\int_{0}^{t}f(s)g(s)ds) \leq C_0C_1 \\
\Longrightarrow \int_{0}^{t}f(s)g(s)ds \leq C_3 , \quad C_3 > 0\\
\Longrightarrow f(t) \leq C_1\cdot (2+C_3) \cdot \ln(2+C_3) \\
\Longrightarrow f(t) \leq M, \quad M = C_1\cdot (2+C_3) \cdot \ln(2+C_3) > 0  
\end{align*}
所以, 存在常数 $M>0$, 使得 $f(t) \leq M, \quad t\in [0,+\infty) \quad \blacksquare$


\section{2017年}

\subsection{2017北京大学数学科学学院秋令营10月13日}
\subsubsection{第3题}
\noindent 3.给定素数$p$, 正整数$n,a$, 其中$(a,p)=1$, 证明: 存在无穷多个正整数k,使得 $p^n|k^k-a$.\\
\textbf{Proof.}\\
对$n$归纳证明. \\
记欧拉函数为$\phi (n)$, 则 : $\phi (p^n) = (p-1)p^{n-1}$ \\
(1) 当$n=1$时, 取$k$满足如下条件: $k \equiv a \pmod p ,\quad	k \equiv 1 \pmod {p-1}$, \\
那么由中国剩余定理解出满足条件的 $k = b + m\cdot p(p-1) , 0<b < p(p-1)$, 其中: 
\begin{eqnarray*}
&	b \equiv a \pmod p ,\quad	b \equiv 1 \pmod {p-1} \\
&	\Longrightarrow a^b \equiv a^{1+h\cdot (p-1)} \equiv a \pmod p
\end{eqnarray*}
从而, 有 $(k,p)=1$:
$$k \equiv a \pmod p \Longrightarrow k^k \equiv a^k \equiv a^b \equiv a \pmod p$$ 
(2) 假设 $n=t$时,命题成立, 即存在$u$, $(u,p)=1$, 并且满足: $u^u \equiv a \pmod {p^t}$, 记 $u^u = l\cdot p^t + a$,\\
当 $n=t+1$时, 取 $k=u + m\cdot \phi (p^{t+1})$, 那么$(k,p)=1$, 从而$(k,p^{t+1}) =1$, 于是: 
\begin{eqnarray*}
	& k \equiv u+m\cdot \phi(p^{t+1}) \equiv u + m\cdot p^{t+1} - m\cdot p^t \equiv u-m\cdot p^t \pmod{p^{t+1}}\\
	&\Longrightarrow  k^k \equiv (u-m\cdot p^t)^k \equiv (u-m\cdot p^t)^u \equiv u^u -C_u^1\cdot u^{u-1} m\cdot p^t \equiv u^u - u^u\cdot m\cdot p^t \\
	& \equiv u^u(1-m\cdot p^t) \equiv (l\cdot p^t +a)(1-m\cdot p^t) \equiv a + (l-a\cdot m)p^t \pmod{p^{t+1}}
\end{eqnarray*}
因为 $(a,p)=1$, 所以存在$m$, 满足: $l-am=s\cdot p$, \\
并且满足条件的$m$是无穷多个,从而$k$也是无穷多个, 于是:
$$ k^k \equiv a + s\cdot p^{t+1} \equiv a \pmod {p^{t+1}}$$
由归纳假定, 对任意的$n$, 存在无穷多个$k$, $(k,p)=1$ 满足: $p^n|k^k-a$.

\subsection{2017清华大学金秋营10月13号}
\subsubsection{第1题}
\noindent 1. 设T是平面到自身的映射, 对平面上任意两点P,Q, d(P,Q) = d(T(P),T(Q)), 这里d(P,Q)代表两点之间的距离。证明: 存在实数$a,b,c,d,x_0,y_0$ 使得在平面直角坐标系下, T把每个点 $x,y$ 映射成 $(ax+by+x_0, cx+dy+y_0)$。\\
\textbf{Proof.} \\
显然平移和旋转变换是保持距离不变的线性映射, 设旋转角为 $\theta$, 平移的距离为$(x_0,y_0)$ 于是由点$(x,y)$到点$(u,v)$的复合变换可以表示为: \\
$$
\begin{bmatrix}
u \\
v
\end{bmatrix} =
\begin{bmatrix}
\cos \theta & -\sin \theta \\
\sin \theta & \cos \theta 
\end{bmatrix} \cdot 
\begin{bmatrix}
x \\
y
\end{bmatrix} + 
\begin{bmatrix}
x_0 \\
y_0
\end{bmatrix}
$$
所以, 存在实数 $a=\cos \theta, b=-\sin \theta, c=\sin \theta, d=\cos \theta, x_0,y_0$, 其中 $\theta, x_0,y_0$ 为任意实数,构成的上述映射T满足要求。

\subsubsection{第2题}
\noindent 2.设连续函数 $f: (0,+\infty) \longrightarrow (0,+\infty)$, 满足: $f(\frac{x+y}{2}) = f(\sqrt{xy}); (x,y>0)$ , 求出所有满足条件的 $f(x)$, 并证明。\\
\textbf{Solution.} \\
(1) 令$y=\frac{1}{x}$, 于是:
$$f\left (\frac{1}{2}(x+\frac{1}{x})\right) = f(1)$$
记: 
$$z=\frac{1}{2}(x+\frac{1}{x}) \ge 1 \Longrightarrow f(z) = f(1) $$
(2) 令$x=1+a,y=1-a , 0<a<1$,则有:
$$f(1) = f(\sqrt{1-a^2})$$
记:  $z=\sqrt{1-a^2} \in (0,1)$, 则有:
$$f(z) = f(1)$$

\noindent \textbf{Or: }(2) 令$x=2\cos^2\theta, y=2\sin^2\theta$, 其中 $\theta \in (0,\frac{\pi}{2})$, 则有:
$$f(1) = f(\sqrt{4\cos^2\theta \cdot \sin^2\theta}) = f(2\cos\theta \cdot \sin \theta) = f(\sin 2\theta)$$
记: $z=\sin2\theta \in (0,1)$, 则有: $f(z) =1$
\\
\\
综上, 得到: $f(x)=f(1) = \text{常数}, x\in (0,+\infty).$
\\
\\
\noindent \textbf{Or:} 给定任意的 $a>b>0$, 令:
$$ x = a - \sqrt{a^2-b^2}, y = a + \sqrt{a^2-b^2}$$
则:
$$ f(a) = f(\frac{x+y}{2}) = f(\sqrt{xy}) = f(b) $$
所以, 对任意的 $x>0$, 有: $f(x) = $常数。 $\blacksquare$


\subsubsection{第3题}
\noindent 3.设 $m,n\in N, m \le n$, 证明: $\sum_{i=0}^{m}C_n^i < \left(\frac{3n}{m}\right)^m$
\\
\textbf{证:}\\
\begin{eqnarray*}
C_n^i=\frac{n!}{i!(n-i)!} = \frac{n(n-1)\cdots(n-i+1)}{i!} < \frac{n^i}{i!} = \frac{m^i}{i!}\cdot \left(\frac{n}{m}\right)^i < \left( \frac{n}{m}\right)^m \cdot \frac{m^i}{i!} \\
\\
\sum_{i=0}^{m}C_n^i < \sum_{i=0}^{m}\left( \frac{n}{m}\right)^m \cdot \frac{m^i}{i!} = \left( \frac{n}{m}\right)^m \cdot \sum_{i=0}^{m}\frac{m^i}{i!} < \left( \frac{n}{m}\right)^m \cdot e^m = \left( \frac{e\cdot n}{m}\right)^m < \left( \frac{3n}{m}\right)^m
\end{eqnarray*}

显然有: $$\sum_{i=0}^{m}C_n^i  = \sum_{i=0}^{m}C_{n-1}^i + \sum_{i=0}^{m-1}C_{n-1}^i $$ \\
当$m=n$时, 不等式显然成立。 下面假设 $1 \le m<n$. 对$n$进行归纳证明。\\
(1) 当$n=2$时, 有$m=1$, 不等式显然成立。 \\
(2) 假设当$n=k$时, 不等式 $\sum_{i=0}^{m}C_k^i < \left(\frac{3k}{m}\right)^k$ 成立, 则当$n=k+1$时, 有:\\
\begin{eqnarray*}
	\sum_{i=0}^{m}C_{k+1}^i = C_{k+1}^0 + \sum_{i=1}^{m}(C_k^i + C_k^{i-1}) = \sum_{i=0}^{m}C_k^i + \sum_{i=0}^{m-1}C_k^i < \left(\frac{3k}{m}\right)^k + \left(\frac{3k}{m-1}\right)^k \\
	\Longleftrightarrow \left(\frac{3k}{m}\right)^k + \left(\frac{3k}{m-1}\right)^k < \left(\frac{3(k+1)}{m}\right)^{k+1} = \frac{3(k+1)}{m} \cdot \left(\frac{3(k+1)}{m}\right)^{k}	\\
	\Longleftrightarrow 1+\left(\frac{m}{m-1}\right)^k < \frac{3(k+1)}{m} \left(\frac{k+1}{k}\right)^{k}
\end{eqnarray*}

\subsubsection{第5题}
\noindent 5. 给定奇素数$p$, 求集合$\left\{(x,y)\mid x^2+y^2 \equiv a \pmod{p}; x,y\in \{0,1,2,\cdots ,(p-1)\} \right\}$ 的元素个数。
\\
\textbf{Solution.} \\
首先, $a \ne 0 \pmod p$时, 方程$uv\equiv a \pmod p $的解$(u,v)$的数目是 $p-1$,为简单记,假定$a\in\{0,1,2,...,p-1\}$ .\\
记 某个元$b$模$p$的逆元$c=b^{-1}$, 也即$bc\equiv b\cdot b^{-1}\equiv1\pmod{p}$\\
(1) 当$p\equiv1 \pmod{4}$时,由欧拉判别法,若$b$是模$p$的平方剩余, 则$-b$也是模$p$的平方剩余, 于是平方剩余是以$\pm b$的形式成对出现。从而$-1$是模p的平方剩余,可记$z^2\equiv -1 \pmod{p}$, 

(i) 如果$a = 0 $, 那么$x^2\equiv -y^2 \pmod{p}$, 由于平方剩余以$\pm b$形式成对出现,因此$x=y=0$,或者$x,y\ne 0$以$2(p-1)$组出现,
此时解的总数为: $1+2(p-1)=2p-1$

(ii) 如果$a\ne 0 $, 那么有:
\begin{eqnarray*}
	&x^2+y^2 \equiv a \pmod{p} & \Longrightarrow x^2 - (-1)y^2 \equiv a \pmod{p}\\
	&\Longrightarrow x^2 - (zy)^2 \equiv a \pmod{p} & \Longrightarrow (x-zy)(x+zy) \equiv a \pmod{p}\\
	&\Longrightarrow u\cdot v \equiv a \pmod{p}
\end{eqnarray*}
其中,$(u,v)$ 和 $(x,y)$一一对应.
\begin{eqnarray*}
	u=x-zy,v=x+zy \\
	\Longleftrightarrow x=(u+v)2^{-1}, y=(u-v)2^{-1}z^{-1}
\end{eqnarray*}
不同的$(u,v)$一共有$p-1$组, 所以不同的解$(x,y)$的数目也是 $p-1$.
\\
(2) 当$p\equiv3 \pmod{4}$时,由欧拉判别法, 如果$b$是平方剩余, 那么$-b$是平方非剩余,

(i) 如果$a = 0 $, 那么$x^2+y^2 \equiv 0 \pmod p$,只有一组解: $(x,y)=(0,0)$

(ii) 如果$a\ne 0 $, 当$y$ 依次取遍 $\{0,1,2,...,p-1\}$时, 考虑$x$的可能取值时, 对每个$y$,下述方程只有一个有解:\\
\begin{eqnarray*}
	x^2 \equiv a - y^2 \pmod p \\
	x^2 \equiv y^2 - a \pmod p \\
\end{eqnarray*}
因为所有的$y$的数目是$p$, 所以上述两个方程给出的所有解$(x,y)$的数目是$2p$, 但是方程:\\
$$x^2 \equiv y^2 - a \pmod p \Longleftrightarrow x^2 - y^2 \equiv - a \equiv b\pmod p \Longleftrightarrow (x+y)(x-y) \equiv b \pmod p$$
解的数目是 $p-1$组, 故 $x^2 \equiv a - y^2 \pmod p$,也即$x^2 +  y^2\equiv a  \pmod p$ 的解的数目是: $2p-(p-1) = p+1$ 。\\
注: 如果$y^2 \equiv a \pmod p$,那么两个方程同时给出解$x=0$, 此时不同的$y$只有2个,所以总的解数目还是$2p$。\\
\\
综上, 解数目$(0 \le a < p-1)$:
$$ =\left\{
\begin{aligned}
2p-1, & \quad a = 0 , p \equiv 1 \pmod 4 \\
1, &\quad  a = 0 , p \equiv 3 \pmod 4 \\
p-1, &\quad  a\ne 0 , p \equiv 1 \pmod 4 \\
p+1, &\quad  a\ne 0 , p \equiv 3 \pmod 4 \\
\end{aligned}
\right.
$$

\section{Conclusion}	
	这是一个CTEX的utf-8编码例子,{\kaishu 这里是楷体显示},{\songti 这里是宋体显示},{\heiti 这里是黑体显示},{\fangsong 这里是仿宋显示},{\lishu 这里是隶书显示},{\youyuan 这里是幼圆显示}。
\end{document}